%-----------------------------------------------------------------
%Di sini awal masukan untuk Prakata
%-----------------------------------------------------------------
\preface
\justifying
\noindent Puji syukur penulis panjatkan ke hadirat Allah SWT atas limpahan rahmat dan karunia-Nya, sehingga penulis dapat menyelesaikan skripsi ini dengan baik. Skripsi ini disusun sebagai salah satu syarat untuk memperoleh gelar Sarjana pada Program Studi Sains Data, Institut Teknologi Sumatera. Dalam proses penyusunan skripsi ini, penulis menerima banyak dukungan, bimbingan, dan bantuan dari berbagai pihak. Oleh karena itu, dengan penuh rasa hormat dan kerendahan hati, penulis menyampaikan ucapan terima kasih yang sebesar-besarnya kepada:

\begin{enumerate}
\item Bapak Tirta Setiawan, S.Pd., M.Si. selaku Koordinator Program Studi Sains Data Institut Teknologi Sumatera,
\item Bapak Christyan Tamaro Nadeak, M.Si. selaku dosen pembimbing pertama, atas bimbingan, arahan, dan motivasi yang berharga selama proses penelitian ini,
\item Ibu Luluk Muthoharoh, M.Si. selaku dosen pembimbing kedua, atas kesabaran dan dukungan yang senantiasa diberikan,
\item Ibu Dalima dan Bapak Syaiun, yang telah memberikan doa yang tiada henti, kasih sayang yang tak ternilai, serta dukungan moral dan material dalam setiap langkah penulis,
\item Teman-teman seperjuangan di kontrakan Tidurlah, atas canda tawa dan semangat selama masa studi,
\item Rekan-rekan dari Program Studi Sains Data angkatan 2020, yang telah menjadi bagian penting dalam perjalanan akademik ini. 
\end{enumerate}

Penulis menyadari bahwa penyusunan Skripsi ini jauh dari sempurna.
Akhir kata penulis mohon maaf yang sebesar-besarnya apabila ada kekeliruan di dalam penulisan skripsi ini.


\vspace{0.5cm}

\begin{flushright}
\begin{tabular}{p{7.5cm}l}
&Lampung Selatan, \approvaldatenc \\[2.5cm]
&\textbf{\fullnamenc}
\end{tabular}
\end{flushright}
