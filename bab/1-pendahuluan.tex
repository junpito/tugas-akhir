\chapter{PENDAHULUAN}
\section{Latar Belakang}
\label{section:latarbelakang}
\noindent Ginjal, organ vital dalam tubuh manusia, memiliki jaringan pembuluh darah kecil yang dikenal sebagai struktur mikrovaskular. Jaringan ini, dengan diameter kurang dari 100 mikrometer, memainkan peran penting dalam berbagai fungsi ginjal, seperti penyaringan darah, pengaturan tekanan darah, dan pengaturan keseimbangan elektrolit \cite{hu_multi-scale_2023}. Pemetaan struktur mikrovaskular secara detail menjadi kunci untuk memahami bagaimana sel-sel ginjal berinteraksi satu sama lain dan untuk mempelajari berbagai penyakit ginjal.
Upaya pemetaan sel-sel manusia secara komprehensif sedang dilakukan oleh Human Cell Atlas (HCA) dari Chan Zuckerberg Initiative, Human Protein Atlas (HPA) dari Knut and Allice Wallenberg Foundation, dan Human BioMolecular Atlas Program (HuBMAP) dari National Institutes of Health (NIH) \cite{weber_considerations_2020}. Proyek-proyek ambisius ini menggunakan pembuluh darah, termasuk mikrovaskular, sebagai sistem navigasi utama untuk memetakan seluruh sel-sel sehat di seluruh tubuh manusia, termasuk ginjal.

\noindent Meskipun proses pemetaan mikrovaskular secara manual menawarkan presisi tinggi, proses ini memakan waktu lama dan membutuhkan keahlian khusus dalam bidang patologi dan analisis citra medis \cite{hu_multi-scale_2023,weber_considerations_2020}. Hal ini disebabkan oleh kompleksitas struktur mikrovaskular dan kerumitan interpretasi citra mikroskopis. Tantangan ini diperparah dengan ketersediaan data citra mikroskopis ginjal yang sangat besar, yang membutuhkan waktu dan tenaga kerja yang signifikan untuk dianalisis secara manual. Oleh karena itu, penelitian ini bertujuan untuk mengembangkan metode otomatis untuk segmentasi mikrovaskular ginjal dari whole slide image (WSI) menggunakan algoritma deep learning Fully Convolutional Network (FCN). Algoritma FCN terbukti efektif dalam menganalisis citra medis dan mengidentifikasi objek dengan presisi tinggi \cite{huang_fully_2022}. Diharapkan metode ini dapat meningkatkan akurasi dan efisiensi pemetaan mikrovaskular ginjal secara signifikan, sehingga dapat mempercepat penelitian di bidang kesehatan ginjal dan membuka jalan bagi diagnosis dan terapi penyakit ginjal yang lebih presisi. 
%Berbeda dengan penelitian sebelumnya yang menggunakan metode segmentasi tradisional berbasis morfologi atau thresholding, penelitian ini memanfaatkan kekuatan deep learning untuk mempelajari pola kompleks struktur mikrovaskular ginjal dari WSI. Algoritma FCN akan dilatih dengan menggunakan kumpulan data WSI ginjal yang telah dilabel dengan anotasi mikrovaskular yang akurat. Model FCN yang dihasilkan kemudian akan digunakan untuk segmentasi mikrovaskular secara otomatis pada WSI ginjal baru, menghasilkan peta mikrovaskular yang terperinci dan akurat.

\section{Rumusan Masalah}
\noindent Bagian ini menjadi salah satu bagian penting dalam Pendahuluan. Setelah paparan Latar Belakang \ref{section:latarbelakang}, maka masalah yang diangkat pada pekerjaan penelitian perlu dirumuskan dengan baik. Pertanyaan apa yang akan dijawab dalam penelitian dapat ditulis dalam kalimat tanya ataupun tidak.

\noindent Berdasarkan latar belakang yang telah dijelaskan sebelumnya, berikut merupakan rumusan masalah pada penelitian tugas akhir ini:
\begin{enumerate}
    \item Bagaimana curah hujan berhubungan dengan tingkat pendapatan masyarakat?
    \item Apakah Candi Borobudur dibangun dengan mengikuti kaidah astronomi?
\end{enumerate}

\section{Tujuan Penelitian}
\noindent Eros reprimique vim no. Alii legendos volutpat in sed, sit enim nemore labores no. No odio decore causae has. Vim te falli libris neglegentur, eam in tempor delectus dignissim, nam hinc dictas an.

\noindent Tujuan dari penelitian ini berdasarkan rumusan masalah yang juga menjadi dasar dilakukannya penelitian ini adalah sebagai berikut:
\begin{enumerate}
    \item Melihat hubungan curah hujan dengan tingkat pendapatan masyarakat dengan metode blabla?
    \item Melakukan simulasi langit malam di masa Candi Borobudur dibangun?
\end{enumerate}

\section{Batasan Masalah}
\noindent Setiap masalah dan penelitian yang diangkat selalu memiliki batasan. Ada batasan, asumsi, atau kriteria yang menjadi pembatas atas masalah yang diangkat dalam penelitian TA, sehingga arah penelitian dapat fokus. Batasan ini perlu dituliskan secara tegas, dan dapat saja memuat lebih dari satu. Contoh batasan masalah misalnya batasan penggunaan data, area, rentang waktu, dan lain-lain.

% Sub bab lain dapat ditambahkan, misalnya:
%\section{Manfaat Penelitian}

@Article{ID,
	author = {author},
	title = {title},
	journal = {journal},
	year = {year},
}
%\section{Hipotesis}