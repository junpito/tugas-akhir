\chapter{PENDAHULUAN}
\section{Latar Belakang}
\label{section:latarbelakang}
\noindent Bagian ini mendeskripsikan gambaran umum, konteks, dan posisi penelitian skripsi dalam konstelasi perkembangan pengetahuan yang telah dicapai. Penjelasan yang dituliskan menjadi penting karena dengan landasan yang kuat, maka pekerjaan penelitian yang lebih terarah dapat dilakukan. Hal ini lebih spesifik dan tegas disampaikan pada sub-sub bab berikutnya.

\noindent Beberapa pustaka utama yang berperan dominan dapat disampaikan di sini untuk memberi gambaran tentang letak penelitian TA dalam konstelasi keilmuan yang dicapai. Hasil-hasil dari pustaka terbaru dapat menopang Latar Belakang ini menjadi lebih kuat. Sangat wajar apabila isi sub bab setelah Latar Belakang ini mengalami penyesuaian saat sejumlah hasil penelitian sudah diperoleh dan dianalisis. Oleh karena itu, finalisasi isi Pendahuluan ini biasanya dilakukan menjelang akhir pembuatan laporan penelitian yang dituangkan dalam skripsi \cite{malasan_photometric_1986, malasan_remote_2006}.

\section{Rumusan Masalah}
\noindent Bagian ini menjadi salah satu bagian penting dalam Pendahuluan. Setelah paparan Latar Belakang \ref{section:latarbelakang}, maka masalah yang diangkat pada pekerjaan penelitian perlu dirumuskan dengan baik. Pertanyaan apa yang akan dijawab dalam penelitian dapat ditulis dalam kalimat tanya ataupun tidak.

\noindent Berdasarkan latar belakang yang telah dijelaskan sebelumnya, berikut merupakan rumusan masalah pada penelitian tugas akhir ini:
\begin{enumerate}
    \item Bagaimana curah hujan berhubungan dengan tingkat pendapatan masyarakat?
    \item Apakah Candi Borobudur dibangun dengan mengikuti kaidah astronomi?
\end{enumerate}

\section{Tujuan Penelitian}
\noindent Eros reprimique vim no. Alii legendos volutpat in sed, sit enim nemore labores no. No odio decore causae has. Vim te falli libris neglegentur, eam in tempor delectus dignissim, nam hinc dictas an.

\noindent Tujuan dari penelitian ini berdasarkan rumusan masalah yang juga menjadi dasar dilakukannya penelitian ini adalah sebagai berikut:
\begin{enumerate}
    \item Melihat hubungan curah hujan dengan tingkat pendapatan masyarakat dengan metode blabla?
    \item Melakukan simulasi langit malam di masa Candi Borobudur dibangun?
\end{enumerate}

\section{Batasan Masalah}
\noindent Setiap masalah dan penelitian yang diangkat selalu memiliki batasan. Ada batasan, asumsi, atau kriteria yang menjadi pembatas atas masalah yang diangkat dalam penelitian TA, sehingga arah penelitian dapat fokus. Batasan ini perlu dituliskan secara tegas, dan dapat saja memuat lebih dari satu. Contoh batasan masalah misalnya batasan penggunaan data, area, rentang waktu, dan lain-lain.

% Sub bab lain dapat ditambahkan, misalnya:
%\section{Manfaat Penelitian}
%\section{Hipotesis}