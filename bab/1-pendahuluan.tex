\chapter{PENDAHULUAN}
\section{Latar Belakang}
\label{section:latarbelakang}
\noindent Ginjal, organ vital dalam tubuh manusia, memiliki jaringan pembuluh darah kecil yang dikenal sebagai struktur mikrovaskular. Jaringan ini, dengan diameter kurang dari 100 mikrometer, memainkan peran penting dalam berbagai fungsi ginjal, seperti penyaringan darah, pengaturan tekanan darah, dan pengaturan keseimbangan elektrolit \cite{hu_multi-scale_2023}. Pemetaan struktur mikrovaskular secara detail menjadi kunci untuk memahami bagaimana sel-sel ginjal berinteraksi satu sama lain dan untuk mempelajari berbagai penyakit ginjal.
Upaya pemetaan sel-sel manusia secara komprehensif sedang dilakukan oleh Human Cell Atlas (HCA) dari Chan Zuckerberg Initiative, Human Protein Atlas (HPA) dari Knut and Allice Wallenberg Foundation, dan Human BioMolecular Atlas Program (HuBMAP) dari National Institutes of Health (NIH) \cite{weber_considerations_2020}. Proyek-proyek ambisius ini menggunakan pembuluh darah, termasuk mikrovaskular, sebagai sistem navigasi utama untuk memetakan seluruh sel-sel sehat di seluruh tubuh manusia, termasuk ginjal.

\noindent Meskipun proses pemetaan mikrovaskular secara manual menawarkan presisi tinggi, proses ini memakan waktu lama dan membutuhkan keahlian khusus dalam bidang histologi dan analisis citra biomedis \cite{hu_multi-scale_2023,weber_considerations_2020}. Hal ini disebabkan oleh kompleksitas struktur mikrovaskular dan kerumitan interpretasi citra mikroskopis. Tantangan ini diperparah dengan ketersediaan data citra mikroskopis ginjal yang sangat besar, yang membutuhkan waktu dan tenaga kerja yang signifikan untuk dianalisis secara manual. Oleh karena itu, penelitian ini bertujuan untuk mengembangkan metode otomatis untuk segmentasi mikrovaskular ginjal dari whole slide image (WSI) menggunakan algoritma deep learning Fully Convolutional Network (FCN). Algoritma FCN terbukti efektif dalam menganalisis citra medis dan mengidentifikasi objek dengan presisi tinggi \cite{huang_fully_2022}. Diharapkan metode ini dapat meningkatkan akurasi dan efisiensi pemetaan mikrovaskular ginjal secara signifikan, sehingga dapat mempercepat penelitian di bidang kesehatan ginjal dan membuka jalan bagi diagnosis dan terapi penyakit ginjal yang lebih presisi. 

\noindent Penelitian terdahulu telah menunjukkan potensi penggunaan FCN dalam segmentasi citra medis di berbagai area anatomis. Yosefi et al. (2021) meneliti penggunaan FCN modifikasi ("DDAUnet") pada segmentasi tumor esofagus dari data CT, dengan hasil rata-rata DSC 0.79. Penelitian ini menunjukkan potensi FCN dalam segmentasi tumor esofagus \cite{yousefi_esophageal_2021}. Selanjutnya, Chai et al. (2020) menerapkan arsitektur "MA-Unet" yang dimodifikasi pada segmentasi citra tomografi terkomputasi (CT scan) 2D untuk mengidentifikasi struktur paru-paru, mencapai hasil akurasi mIoU 0.96. Penelitian ini menunjukkan efektivitas FCN dalam segmentasi struktur paru-paru \cite{cai_ma-unet_2020}. Kemudian, Deng et al. (2023) mengembangkan Omni-Seg, sebuah jaringan dinamis dengan arsitektur "class-aware" yang dilatih secara khusus pada data multi-label dan sebagian berlabel (partially-labeled) untuk segmentasi berbagai struktur patologis ginjal pada WSI. Mereka menyoroti kemampuan Omni-Seg menangani beragam jenis citra, menghasilkan rata-rata Dice similarity co-
efficien (mean DSC) 87.70 \cite{deng_omni-seg_2022}. Penelitian ini menunjukkan bahwa pendekatan yang fleksibel mungkin efektif untuk masalah segmentasi mikrovaskular ginjal di WSI.

\noindent Penelitian terdahulu telah menunjukkan potensi FCN terutama arsitektur U-net dalam segmentasi citra biomedis dari berbagai area anatomi, namun masih terdapat keterbatasan dalam menangani kompleksitas struktur mikrovaskular dan keragaman data WSI. Penelitian ini akan dilakukan evaluasi terhadap Attention U-net yang dilatih menggunakan data mikrovaskular dari Human BioMolecular Atlas Program (HubMap). Data yang akan digunakan telah dianotasi oleh para ahli di bidang patologi ginjal \cite{howard_hubmap_2023}. Penggunaan data yang telah divalidasi ini diharapkan dapat meningkatkan akurasi dan presisi model dalam segmentasi struktur mikrovaskular ginjal. Dengan demikian, penelitian ini memiliki potensi untuk memberikan kontribusi dalam pemetaan mikrovaskular ginjal dan membuka jalan bagi penelitian yang lebih mendalam tentang fungsi ginjal dan penyakit ginjal. 

\section{Rumusan Masalah}
\noindent Berdasarkan latar belakang yang telah dijelaskan sebelumnya, berikut merupakan rumusan masalah pada penelitian tugas akhir ini:
\begin{enumerate}
    \item Apakah FCN terutama arsitektur Attention U-net dapat digunakan untuk segmentasi mikrovaskular di WSI jaringan ginjal manusia dengan akurasi yang tinggi?
    \item Bagaimana pengaruh pengaruh penambahan modul attention gate pada arsitektur U-net untuk segmentasi mikrovaskular ginjal?
    \item Jenis FCN mana yang memiliki performa terbaik untuk segmentasi mikrovaskular ginjal?
\end{enumerate}

\section{Tujuan Penelitian}

\noindent Tujuan dari penelitian ini berdasarkan rumusan masalah yang juga menjadi dasar dilakukannya penelitian ini adalah sebagai berikut:
\begin{enumerate}
    \item Menganalisis kinerja FCN dalam segmentasi mikrovaskular di WSI jaringan ginjal manusia.
    \item Mengevaluasi pengaruh penambahan modul attention gate pada arsitektur U-net untuk segmentasi mikrovaskular ginjal.
    \item Membandingkan performa jenis FCN untuk segmentasi mikrovaskular ginjal.
\end{enumerate}

\section{Batasan Masalah}
\noindent Penelitian ini memiliki batasan masalah yang harus di perhatikan sebagai berikut:
\begin{enumerate}
	\item Penelitian ini terbatas pada penerapan FCN di arsitektur Attention U-net pada segmentasi WSI jaringan mikrovaskular manusia menggunakan dataset yang di sediakan dan dianotasikan oleh HubMap.
	\item Pelatihan model terbatas pada data yang telah dianotasikan pada data yang telah dirilis HuBMAP.
	\item Penerapan attention gate terbatas pada skip connection U-net.
	%\item Pembadingan peforma terbatas pada FCN8s dan U-net dikarenakan terbatasnya waktu dan kurangnya alat komputasi.
\end{enumerate}

% Sub bab lain dapat ditambahkan, misalnya:
%\section{Manfaat Penelitian}
%\section{Hipotesis}