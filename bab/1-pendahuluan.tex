\chapter{PENDAHULUAN}
\section{Latar Belakang}
\label{section:latarbelakang}
\noindent Struktur mikrovaskular, seperti arteroil, venula, dan kapiler, merupakan jaringan pembuluh darah kecil di seluruh tubuh yang memiliki diameter kurang dari 100 mikrometer. Struktur ini berperan penting dalam berbagai fungsi ginjal, seperti penyaringan darah, pengaturan tekanan darah, dan pengaturan keseimbangan elektrolit \cite{hu_multi-scale_2023}. Selain itu, mikrovaskular juga berperan penting sebagai kordinat untuk memetakan seluruh sel sehat yang ada di seluruh tubuh manusia \cite{}. Pemetaan ini penting untuk memahami bagaimana sel-sel sehat berinteraksi satu sama lain dan untuk mempelajari penyakit.


\noindent Meskipun proses pemetaan mikrovaskular secara manual memakan waktu lama dan membutuhkan keahlian khusus dalam bidang patologi dan analisis citra medis, proses ini tetap menjadi langkah penting dalam penelitian kesehatan. Oleh karena itu, penelitian ini bertujuan untuk mengotomatiskan segmentasi mikrovaskular dari whole slide image ginjal manusia menggunakan algoritma deep learning Fully Convolutional Network (FCN). Algoritma FCN adalah algoritma yang dirancang untuk menganalisis citra dan mengidentifikasi objek dengan presisi tinggi. Diharapkan algoritma ini dapat meningkatkan akurasi dan efisiensi pemetaan mikrovaskular, sehingga dapat mempercepat penelitian di bidang kesehatan.

\section{Rumusan Masalah}
\noindent Bagian ini menjadi salah satu bagian penting dalam Pendahuluan. Setelah paparan Latar Belakang \ref{section:latarbelakang}, maka masalah yang diangkat pada pekerjaan penelitian perlu dirumuskan dengan baik. Pertanyaan apa yang akan dijawab dalam penelitian dapat ditulis dalam kalimat tanya ataupun tidak.

\noindent Berdasarkan latar belakang yang telah dijelaskan sebelumnya, berikut merupakan rumusan masalah pada penelitian tugas akhir ini:
\begin{enumerate}
    \item Bagaimana curah hujan berhubungan dengan tingkat pendapatan masyarakat?
    \item Apakah Candi Borobudur dibangun dengan mengikuti kaidah astronomi?
\end{enumerate}

\section{Tujuan Penelitian}
\noindent Eros reprimique vim no. Alii legendos volutpat in sed, sit enim nemore labores no. No odio decore causae has. Vim te falli libris neglegentur, eam in tempor delectus dignissim, nam hinc dictas an.

\noindent Tujuan dari penelitian ini berdasarkan rumusan masalah yang juga menjadi dasar dilakukannya penelitian ini adalah sebagai berikut:
\begin{enumerate}
    \item Melihat hubungan curah hujan dengan tingkat pendapatan masyarakat dengan metode blabla?
    \item Melakukan simulasi langit malam di masa Candi Borobudur dibangun?
\end{enumerate}

\section{Batasan Masalah}
\noindent Setiap masalah dan penelitian yang diangkat selalu memiliki batasan. Ada batasan, asumsi, atau kriteria yang menjadi pembatas atas masalah yang diangkat dalam penelitian TA, sehingga arah penelitian dapat fokus. Batasan ini perlu dituliskan secara tegas, dan dapat saja memuat lebih dari satu. Contoh batasan masalah misalnya batasan penggunaan data, area, rentang waktu, dan lain-lain.

% Sub bab lain dapat ditambahkan, misalnya:
%\section{Manfaat Penelitian}

@Article{ID,
	author = {author},
	title = {title},
	journal = {journal},
	year = {year},
}
%\section{Hipotesis}