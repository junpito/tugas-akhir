\chapter{PENDAHULUAN}
\section{Latar Belakang}
\label{section:latarbelakang}
\noindent Ginjal, organ vital dalam tubuh manusia, memiliki jaringan pembuluh darah kecil yang dikenal sebagai mikrovaskular. Jaringan ini, dengan diameter kurang dari 100 mikrometer, memainkan peran penting dalam berbagai fungsi ginjal, seperti penyaringan darah, pengaturan tekanan darah, dan pengaturan keseimbangan elektrolit \cite{hu_multi-scale_2023}. Segmentasi struktur mikrovaskular secara akurat menjadi kunci untuk memahami bagaimana sel-sel ginjal berinteraksi satu sama lain dan untuk mempelajari berbagai penyakit ginjal \cite{zhao_attention-based_2023}. Hal ini berkaitan juga dengan proyek pemetaan sel-sel manusia secara komprehensif yang dilakukan oleh \textit{Human Cell Atlas} (HCA) oleh Chan Zuckerberg Initiative, \textit{Human Protein Atlas} (HPA) oleh Knut and Allice Wallenberg Foundation, dan \textit{Human BioMolecular Atlas Program} (HuBMAP) oleh National Institutes of Health \cite{weber_considerations_2020}. Proyek-proyek ambisius ini menggunakan kerangka kerja \textit{Vascular Coordinate Common Framework} (VCCF) dimana pembuluh darah, termasuk mikrovaskular digunakan sebagai sistem navigasi utama untuk memetakan seluruh sel-sel sehat di seluruh tubuh manusia, termasuk ginjal. 

\noindent Meskipun proses segmentasi mikrovaskular secara manual menawarkan presisi tinggi, proses ini memakan waktu lama dan membutuhkan keahlian khusus dalam bidang histologi dan analisis citra biomedis \cite{hu_multi-scale_2023,weber_considerations_2020}. Hal ini disebabkan oleh kompleksitas struktur mikrovaskular, ukuran yang beragam dan kerumitan interpretasi citra mikroskopis dari mikrovaskular. Tantangan ini diperparah dengan ketersediaan data citra mikroskopis ginjal yang sangat besar, yang membutuhkan waktu dan tenaga kerja yang signifikan untuk dianalisis secara manual. Kemudian penggunaan teknik komputer vision konvensional seperti metode berbasis fitur juga kesulitan untuk menangkap batas-batas mikrovaskular, karena tekstur halus dan tepi yang lemah dari sistem mikrovaskular \cite{zhao_attention-based_2023}.


%Oleh karena itu, penelitian ini bertujuan untuk mengembangkan metode otomatis untuk segmentasi mikrovaskular ginjal dari \textit{Whole Slide Image} (WSI) menggunakan model \textit{deep learning} \textit{Fully Convolutional Network} (FCN). Model \textit{deep learning} FCN terbukti efektif dalam menganalisis citra medis dan mengidentifikasi objek dengan presisi tinggi \cite{huang_fully_2022}. Diharapkan metode ini dapat meningkatkan akurasi dan efisiensi pemetaan mikrovaskular ginjal secara signifikan, sehingga dapat mempercepat penelitian di bidang kesehatan ginjal dan membuka jalan bagi diagnosis dan terapi penyakit ginjal yang lebih presisi. 

\noindent Untuk menghadapi masalah tersebut penggunaan \textit{deep learning} seperti \textit{fully convolutional network }(FCN) bisa menjadi jawaban. Berbagai penelitian terdahulu menunjukkan potensi penggunaan FCN dalam segmentasi citra medis di berbagai area anatomis. Yosefi et al. (2021) meneliti penggunaan FCN modifikasi ("DDAUnet") pada segmentasi tumor esofagus dari data CT, dengan hasil rata-rata \textit{Dice Similarity Coefficient} (DSC) 0.79. Penelitian ini menunjukkan potensi FCN dalam segmentasi tumor esofagus \cite{yousefi_esophageal_2021}. Chai et al. (2020) menerapkan arsitektur "MA-Unet" yang dimodifikasi pada segmentasi citra tomografi terkomputasi (computed tomography scan) 2D untuk mengidentifikasi struktur paru-paru, mencapai hasil akurasi rata-rata \textit{Intersection over Union} (IoU) 0.96. Penelitian ini menunjukkan efektivitas FCN dalam segmentasi struktur paru-paru \cite{cai_ma-unet_2020}.
 %Kemudian, Deng et al. (2023) mengembangkan Omni-Seg, sebuah jaringan dinamis dengan arsitektur "class-aware" yang dilatih secara khusus pada data multi-label dan sebagian berlabel (partially-labeled) untuk segmentasi berbagai struktur patologis ginjal pada WSI. Mereka menyoroti kemampuan Omni-Seg menangani beragam jenis citra, menghasilkan rata-rata DSC 87.70 \cite{deng_omni-seg_2022}.
Kemudian, Sultan et al. (2023). Telah menginisiasi penelitian terhadap penggunaan arsitektur dasar U-ne untuk segmentasi mikrovaskular. Ekplorasi ini menghasilkan IoU 0.27 dan DSC 0.507 pada Artektur dasar U-net. Kemudian arsitektur \textit{ResNet-50} dan \textit{ResNet-34} juga digunakan sebagai encoder pada U-net di penelitian ini. Menghasilkan IoU 0.408 dan DSC 0.580 pada \textit{ResNet-50}, IoU 0.273 dan DSC 0.430 pada \textit{ResNet-34} \cite{sultan_microvasculature_2023}.


\noindent Penelitian terdahulu telah menunjukkan potensi FCN terutama arsitektur U-net dalam segmentasi citra biomedis dari berbagai area anatomi, namun masih terdapat keterbatasan dalam menangani kompleksitas struktur mikrovaskular. Penelitian ini berfokus pada evaluasi Attention U-net dalam segmentasi mikrovaskular dalam Whole Slie Image  (WSI; citra digital hasil pemindaian seluruh jaringan dengan resolusi sangat tinggi untuk analisis histopatologis) jaringan ginjal. Arsitektur Attention U-net dipilih karena kemampuannya untuk memberikan perhatian lebih pada fitur-fitur penting dalam citra, yang diharapkan dapat meningkatkan akurasi segmentasi struktur mikrovaskular yang kompleks. Model ini akan dilatih menggunakan data mikrovaskular dari HuBMAP, yang telah dianotasi oleh para ahli di bidang histologi ginjal \cite{howard_hubmap_2023}. Penggunaan data yang telah divalidasi ini diharapkan dapat meningkatkan akurasi dan presisi model dalam segmentasi struktur mikrovaskular ginjal. Dengan demikian, penelitian ini memiliki potensi untuk memberikan kontribusi dalam pemetaan mikrovaskular ginjal dan membuka jalan bagi penelitian yang lebih mendalam tentang fungsi ginjal dan penyakit ginjal. 

\section{Rumusan Masalah}
\noindent Berdasarkan latar belakang yang telah dijelaskan sebelumnya, berikut merupakan rumusan masalah pada penelitian tugas akhir ini:
\begin{enumerate}
    \item Bagaimana kinerja model FCN terutama arsitektur Attention U-net, jika digunakan untuk tugas segmentasi mikrovaskular di WSI jaringan ginjal manusia?
    \item Bagaimana pengaruh penambahan modul attention gate pada arsitektur U-net untuk segmentasi mikrovaskular ginjal?
    		%\item Bagaimana perbandingan performa model FCN pada arsitektur dasar FCN dengan arsitektur Attention U-net?
\end{enumerate}

\section{Tujuan Penelitian}

\noindent Tujuan dari penelitian ini berdasarkan rumusan masalah yang juga menjadi dasar dilakukannya penelitian ini adalah sebagai berikut:
\begin{enumerate}
    \item Untuk mengevaluasi kinerja model FCN, terutama arsitektur Attention U-net, dalam tugas segmentasi mikrovaskular pada Whole Slide Images (WSI) jaringan ginjal manusia.
    \item Untuk menganalisis pengaruh penambahan modul attention gate pada arsitektur U-net dalam segmentasi mikrovaskular ginjal.
    %\add{text}\item Untuk membandingkan performa model FCN dengan arsitektur dasar FCN dan arsitektur Attention U-net.
\end{enumerate}

\section{Batasan Masalah}
\noindent Penelitian ini memiliki batasan masalah yang harus di perhatikan sebagai berikut:
\begin{enumerate}
	\item Penelitian ini terbatas pada penerapan model FCN di arsitektur Attention U-net pada segmentasi WSI jaringan mikrovaskular manusia menggunakan dataset yang di sediakan dan telah dianotasikan oleh HuBMAP. Data yang belum dianotasikan atau berasal dari sumber lain tidak akan digunakan.
	\item Penerapan modul attention gate terbatas pada skip connection (jalur pintas yang menghubungkan lapisan encoder dan decoder) U-net.
	\item Penelitian ini berfokus pada pengembangan dan evaluasi model segmentasi mikrovaskular ginjal. Aspek lain seperti analisis fungsional mikrovaskular atau korelasi dengan penyakit ginjal tidak akan dibahas secara mendalam.
	%\item Pembadingan peforma terbatas pada FCN8s dan U-net dikarenakan terbatasnya waktu dan kurangnya alat komputasi.
\end{enumerate}

% Sub bab lain dapat ditambahkan, misalnya:
%\section{Manfaat Penelitian}
%\section{Hipotesis}