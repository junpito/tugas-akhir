\chapter{TINJAUAN PUSTAKA}
% contoh opsi lain Bab 2
%\chapter{DASAR TEORI}


\section{Landasan Teori}
\noindent Penelitian ini didasarkan pada beberapa landasan teori yang relevan dengan segmentasi mikrovaskular ginjal menggunakan Fully Convolutional Network (FCN). Landasan teori ini meliputi pemahaman mendalam tentang mikrovaskular ginjal, konsep whole slide image (WSI), prinsip-prinsip deep learning, dan arsitektur FCN.

\subsection{Mikrovaskular Ginjal}
\noindent Ginjal adalah organ vital yang bertanggung jawab untuk menyaring darah, mengatur tekanan darah, dan menjaga keseimbangan elektrolit dalam tubuh manusia \cite{ito_s-27-1_2023,bagarao_renal_2023}. %+\citeabdulla_biology_2022
 Ginjal melakukan hal tersebut melalui proses-proses seperti filtrasi glomelurus, penyerapan tubulus, dan sekresi tubulus, yang pada akhirnya membentuk urin \cite{auctores_publishing_llc_what_2021}. Selain itu ginjal memainkan peran penting dalam memproduksi hormon seperti eritropioietin, yang meransang produksi darah merah dan renin, yang terlibat dalam pengaturan tekanan darah \cite{scannali_s-22-6_2023}. Kemudian, ginjal membantu mengatur volume cairan tubuh, kandungan elekrolit, dan keasaman, serta membuang produk limbah dan racun dalam tubuh. Secara garis besar, ginjal sangat penting untuk menjaga lingkungan internal tubuh, guna menjamin kondisi yang optimal agar berbagai proses tubuh dapat berfungsi dengan baik.

\noindent

Eros reprimique vim no. Alii legendos volutpat in sed, sit enim nemore labores no. No odio decore causae has. Vim te falli libris neglegentur, eam in tempor delectus dignissim, nam hinc dictas an. Ne per tota mollis suscipit. Ullum labitur vim ut, ea dicit eleifend dissentias sit. Duis praesent expetenda ne sed. Sit et labitur albucius elaboraret. Ceteros efficiantur mei ad. Hendrerit vulputate democritum est at, quem veniam ne has, mea te malis ignota volumus. Eros reprimique vim no. Alii legendos volutpat in sed, sit enim nemore labores no. No odio decore causae has. Vim te falli libris neglegentur, eam in tempor delectus dignissim, nam hinc dictas an.

Terkadang item ditulis tanpa penomoran, hal ini dilakukan untuk menunjukkan sesuatu yang jumlahnya tidak diketahui secara pasti. Penulisan item yang tidak bernomor contohnya adalah sebagai berikut:
\begin{itemize}
    \item Pro omnium incorrupte ea. Elitr eirmod ei qui, ex partem causae disputationi nec. Amet dicant
    \item No vis, eum modo omnes quaeque ad, antiopam evertitur reprehendunt pro ut.
    \item Nulla inermis est ne. Choro insolens mel ne, eos labitur nusquam eu, nec deserunt reformidans ut. His etiam copiosae principes te, sit brute atqui definiebas id.
\end{itemize}

Terkadang item penomoran ingin diubah sesuai format tertentu. Contohnya adalah sebagai berikut:
\begin{enumerate}[a).]
    \item Pro omnium incorrupte ea. Elitr eirmod ei qui, ex partem causae disputationi nec. Amet dicant
    \item No vis, eum modo omnes quaeque ad, antiopam evertitur reprehendunt pro ut.
    \item Nulla inermis est ne. Choro insolens mel ne, eos labitur nusquam eu, nec deserunt reformidans ut. His etiam copiosae principes te, sit brute atqui definiebas id.
\end{enumerate}

Terkadang item penomoran ingin diubah sesuai format tertentu. Contohnya adalah sebagai berikut:
\begin{itemize}
    \item[!!] Pro omnium incorrupte ea. Elitr eirmod ei qui, ex partem causae disputationi nec. Amet dicant
    \item[*] No vis, eum modo omnes quaeque ad, antiopam evertitur reprehendunt pro ut.
    \item[Step 1.] Nulla inermis est ne. Choro insolens mel ne, eos labitur nusquam eu, nec deserunt reformidans ut. His etiam copiosae principes te, sit brute atqui definiebas id.
\end{itemize}


\subsubsection{Menulis Subsubsubbab}
Ne per tota mollis suscipit. Ullum labitur vim ut, ea dicit eleifend dissentias sit. Duis praesent expetenda ne sed. Sit et labitur albucius elaboraret. Ceteros efficiantur mei ad. Hendrerit vulputate democritum est at, quem veniam ne has, mea te malis ignota volumus. Eros reprimique vim no. Alii legendos volutpat in sed, sit enim nemore labores no. No odio decore causae has. Vim te falli libris neglegentur, eam in tempor delectus dignissim, nam hinc dictas an.

\subsubsection{Contoh subsubsubbab lainnya}
Pro omnium incorrupte ea. Elitr eirmod ei qui, ex partem causae disputationi nec. Amet dicant no vis, eum modo omnes quaeque ad, antiopam evertitur reprehendunt pro ut. Nulla inermis est ne. Choro insolens mel ne, eos labitur nusquam eu, nec deserunt reformidans ut. His etiam copiosae principes te, sit brute atqui definiebas id.



\section{Menulis Persamaan}
\noindent Persamaan matematis dapat ditulis dalam berbagai bentuk. Beberapa faktor yang mempengaruhi penulisan antara lain: (1) apakah persamaan tadi perlu diberi nomor atau tidak (2) apakah ada persamaan-persamaan tadi dalam sebuah kelompok (3) atau apakah merupakan bentuk penurunan yang perlu disejajarkan (4) selain itu bisa juga persamaan yang ditulis di dalam teks.
\begin{equation}
E=mc^2
\end{equation}
    dengan $E$ adalah energi, $m$ adalah massa, dan $c$ adalah kecepatan cahaya.

\begin{equation}
\sqrt{x^2+1}
\end{equation}
    dengan $x$ adalah variabel.


\subsection{Persamaan inline}
\noindent Quo no atqui omnesque intellegat, ne nominavi argumentum quo. Eum ei purto oporteat dissentiet, soleat utamur an sit. Et assum dicam interpretaris quo. Cetero alterum ea vel, no possit alterum utroque nec. His fuisset quaestio ad. Has eu tritani incorrupte consequuntur, esse aliquip nec ne.
