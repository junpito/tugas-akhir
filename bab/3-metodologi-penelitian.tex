\chapter{METODE PENELITIAN}

\section{Isi Metode Penelitian}
\noindent Alat dan bahan yang digunakan pada penelitian ini terbagi atas perangkat keras dan perangkat lunak yang akan dijelaskan seperti berikut.

\section{Data}
\noindent Pro omnium incorrupte ea. Elitr eirmod ei qui, ex partem causae disputationi nec. Amet dicant no vis, eum modo omnes quaeque ad, antiopam evertitur reprehendunt pro ut. Nulla inermis est ne. Choro insolens mel ne, eos labitur nusquam eu, nec deserunt reformidans ut. His etiam copiosae principes te, sit brute atqui definiebas id.

\section{Alat dan Bahan}
\noindent Pro omnium incorrupte ea. Elitr eirmod ei qui, ex partem causae disputationi nec. Amet dicant no vis, eum modo omnes quaeque ad, antiopam evertitur reprehendunt pro ut. Nulla inermis est ne. Choro insolens mel ne, eos labitur nusquam eu, nec deserunt reformidans ut. His etiam copiosae principes te, sit brute atqui definiebas id.

    \subsection{Perangkat Keras}
    Pro omnium incorrupte ea. Elitr eirmod ei qui, ex partem causae disputationi nec. Amet dicant no vis, eum modo omnes quaeque ad, antiopam evertitur reprehendunt pro ut. Nulla inermis est ne. Choro insolens mel ne, eos labitur nusquam eu, nec deserunt reformidans ut. His etiam copiosae principes, teleskop yang bagus.

	\subsection{Perangkat Lunak}
    Pro omnium incorrupte ea. Elitr eirmod ei qui, ex partem causae disputationi nec. Amet dicant no vis, eum modo omnes quaeque ad, antiopam evertitur reprehendunt pro ut. Nulla inermis est ne. Choro insolens mel ne, eos labitur nusquam eu, nec deserunt reformidans ut. His etiam copiosae principes te, sit brute atqui definiebas id.
		\begin{enumerate}[a.]
		\itemsep0em
			\item Arduino IDE for Windows,
			\item IRAF,
		\end{enumerate}

\section{Alur Penelitian}
	Consul graeco signiferumque qui id, usu eu summo dicunt voluptatum, nec ne simul perpetua posidonium. Eos ea saepe prodesset signiferumque. No dolore possit est. Mei no justo intellegebat definitiones, vis ferri lorem eripuit ad. Solum tritani scribentur duo ei, his an adipisci intellegat.

\section{Tahapan Pelaksanaan}
	Consul graeco signiferumque qui id, usu eu summo dicunt voluptatum, nec ne simul perpetua posidonium. Eos ea saepe prodesset signiferumque. No dolore possit est. Mei no justo intellegebat definitiones, vis ferri lorem eripuit ad. Solum tritani scribentur duo ei, his an adipisci intellegat.

\section{Jadwal Kegiatan}
	Quo no atqui omnesque intellegat, ne nominavi argumentum quo. Eum ei purto oporteat dissentiet, soleat utamur an sit. Et assum dicam interpretaris quo. Cetero alterum ea vel, no possit alterum utroque nec. His fuisset quaestio ad. Has eu tritani incorrupte consequuntur, esse aliquip nec ne \ref{jadwal}.


	%Format tabel ini hanya untuk jadwal kegiatan saja
	%Bisa dibuat dulu di https://www.tablesgenerator.com/ baru dikopi ke dokumen ini
		\begin{table}[H]
		\centering
		\caption{Jadwal Penelitian.} % caption pakai titik?
		\label{jadwal}
		\begin{tabular}{|c|l|l|l|l|l|l|l|}
		\hline
		\multirow{2}{*}{No} & \multirow{2}{*}{Keterangan} & \multicolumn{6}{c|}{Bulan}                                                                                                                          \\ \cline{3-8}
		                    &                             & 1 & 2 & 3 & 4 & 5 & 6 \\ \hline
		1                   & Studi literatur                                  &\cellcolor{gray} &\cellcolor{gray}&                        &                        &                        &                         \\ \hline
		2                   & Desain                                           &                        &\cellcolor{gray}&\cellcolor{gray}&                        &                        &                         \\ \hline
		3                   & Pembelian bahan                                  &                        &                        &\cellcolor{gray}&                        &                        &                         \\ \hline
		4                   & Pembuatan prototipe                              &                        &                        &\cellcolor{gray}&\cellcolor{gray}&\cellcolor{gray}&                         \\ \hline
		5                   & Uji coba dan perbaikan                           &                        &                        &                        &\cellcolor{gray}&\cellcolor{gray}&                         \\ \hline
		6                   & Penulisan tugas akhir                               &                        &                        &                        &                        &                        &\cellcolor{gray}\\ \hline
		\end{tabular}
		\end{table}


\section{Melampirkan Gambar}
\noindent Pro omnium incorrupte ea. Elitr eirmod ei qui, ex partem causae disputationi nec. Amet dicant no vis, eum modo omnes quaeque ad, antiopam evertitur reprehendunt pro ut. Nulla inermis est ne. Choro insolens mel ne, eos labitur nusquam eu, nec deserunt reformidans ut. His etiam copiosae principes te, sit brute atqui definiebas id.

\begin{figure}[H]
    \centering
    \includegraphics[scale=.2]{gambar/contoh-gambar-miring.png}
    \caption{Komputer}
    \label{fig:komputer}
\end{figure}


\section{Program}
\noindent Untuk melampirkan potongan program dapat menggunakan listing, sebagai contoh

\begin{lstlisting}[language=Python,caption={Program perhitungan bilangan prima}, label={lst:bilanganprima}]
import numpy as np

def incmatrix(genl1,genl2):
    m = len(genl1)
    n = len(genl2)
    M = None #to become the incidence matrix
    VT = np.zeros((n*m,1), int)  #dummy variable

    #compute the bitwise xor matrix
    M1 = bitxormatrix(genl1)
    M2 = np.triu(bitxormatrix(genl2),1)

    for i in range(m-1):
        for j in range(i+1, m):
            [r,c] = np.where(M2 == M1[i,j])
            for k in range(len(r)):
                VT[(i)*n + r[k]] = 1;
                VT[(i)*n + c[k]] = 1;
                VT[(j)*n + r[k]] = 1;
                VT[(j)*n + c[k]] = 1;

                if M is None:
                    M = np.copy(VT)
                else:
                    M = np.concatenate((M, VT), 1)

                VT = np.zeros((n*m,1), int)

    return M
\end{lstlisting}



% Baris ini digunakan untuk membantu dalam melakukan sitasi
% Karena diapit dengan comment, maka baris ini akan diabaikan
% oleh compiler LaTeX.
\begin{comment}
\bibliography{daftar-pustaka}
\end{comment}
