\chapter{KESIMPULAN DAN SARAN}

\section{Kesimpulan}
\noindent Penelitian ini bertujuan untuk mengevaluasi kinerja model Fully Convolutional Network (FCN), khususnya Attention U-net, untuk tugas segmentasi mikrovaskular pada Whole Slide Images (WSI) jaringan ginjal manusia. Selain itu penelitian ini juga membahas pengaruh penambahan attention gate dalam meningkatkan peforma segmentasi.

\noindent Dalam pengembangannya model Attention U-net mencapai peforma tertinggi dengan konfigurasi terbaiknya menggunakan optimizer Adam dengan batch size 4. Selain itu dengan penggunaan teknik oversampling dalam pengembangannya, model attention U-net menghasilkan peningkatan yang signifikan pada DSC (dari 0.560 menjadi 0.624) dan IoU (dari 0.413 menjadi 0.481).  Dibandingkan dengan U-net standar, mekanisme attention terbukti mampu meningkatkan kinerja U-net berdasarkan hasil evaluasi pada metriks utama.

\noindent Hasil ini menunjukkan efektivitas Attention U-Net dalam tugas segmentasi mikrovaskular ginjal, terutama pada pembuluh kecil yang disebabkan oleh mekanisme attention yang membantu model memprioritaskan area yang relevan pada citra input, sehingga mengurangi gangguan dari area latar belakang yang kurang signifikan. Namun, tantangan terbesar model adalah mendeteksi struktur pembuluh yang lebih kompleks dan berukuran besar. Hal ini disebabkan oelh kurangnya dataset yang memiliki pembuluh darah berrukuran besar.

\section{Saran}

Berdasarkan hasil penelitian ini, terdapat beberapa aspek yang perlu dikembangkan untuk memperluas manfaat dan akurasi model Attention U-net dalam segmentasi mikrovaskular ginjal. Penelitian selanjutnya disarankan menambah cakupan dataset WSI jaringan ginjal untuk meningkatkan generalisasi model. Selain itu, disarankan juga untuk melakukan ekplorasi pada penerapan transfer learning pada encoder arsitektur U-net. Penambahan modul hybrid, seperti Vision Transformer, juga dapat menjadi pilihan untuk meningkatkan hasil segmentasi secara keseluruhan.
%\begin{enumerate}
%	\item Lorem ipsum is a pseudo-Latin text used in web design, typography, layout, and printing in place of English to emphasise design elements over content.

%	\item It's also called placeholder (or filler) text. It's a convenient tool for mock-ups.

%	\item It helps to outline the visual elements of a document or presentation, eg typography, font, or layout. Lorem ipsum is mostly a part of a Latin text by the classical author and philospher Cicero.

%	\item Its words and letters have been changed by addition or removal, so to deliberately render its content nonsensical; it's not genuine, correct, or comprehensible Latin anymore.
%\end{enumerate}


%\section{Saran}
%\noindent Hal-hal penting terkait pelaksanaan penelitian yang perlu diperhatikan kedepannya adalah
%\begin{enumerate}
%	\item Lorem ipsum is a pseudo-Latin text used in web design, typography, layout, and printing in place of English to emphasise design elements over content.

%	\item It's also called placeholder (or filler) text. It's a convenient tool for mock-ups.

%	\item It helps to outline the visual elements of a document or presentation, eg typography, font, or layout. Lorem ipsum is mostly a part of a Latin text by the classical author and philospher Cicero.

%	\item Its words and letters have been changed by addition or removal, so to deliberately render its content nonsensical; it's not genuine, correct, or comprehensible Latin anymore.
%\end{enumerate}


% Baris ini digunakan untuk membantu dalam melakukan sitasi
% Karena diapit dengan comment, maka baris ini akan diabaikan
% oleh compiler LaTeX.
\begin{comment}
\bibliography{daftar-pustaka}
\end{comment}
