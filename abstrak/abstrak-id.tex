\begin{abstractind}
\justifying

\noindent Struktur mikrovaskular dalam ginjal manusia memainkan peran penting dalam fungsi tubuh, namun segmentasi secara manual memerlukan waktu yang lama dan keahlian khusus dalam histologi. Kompleksitas struktur mikrovaskular menjadi tantangan utama dalam pengembangan metode segmentasi otomatis pada citra histologi.. Penelitian ini mengevaluasi model Attention U-Net untuk segmentasi mikrovaskular pada Whole Slide Image (WSI) jaringan ginjal manusia dengan penambahan modul attention gate pada arsitektur U-Net guna meningkatkan akurasi. Dataset yang digunakan berasal dari Human BioMolecular Atlas Program (HuBMAP) dan telah dianotasi oleh ahli histologi. Model dilatih menggunakan optimizer Adam dengan batch size 4 untuk mencapai performa terbaik, kemudian dievaluasi menggunakan metrik Dice Similarity Coefficient (DSC), Intersection over Union (IoU), precision, dan recall. Hasil penelitian menunjukkan bahwa Attention U-Net unggul dibandingkan U-Net standar pada seluruh metriks utama. Mekanisme attention gate membantu model memfokuskan pada area relevan, sehingga meningkatkan kemampuan model dalam deteksi pembuluh darah kecil.

\bigskip
\noindent
\textbf{Kata kunci:} \textit{Attention U-net, Deep Learning, Dice Similarity Coefficient (DSC), Intersection over Union (IoU), Segmentasi Mikrovaskular}  % masukkan keyword
\end{abstractind}
