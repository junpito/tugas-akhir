\begin{abstractind}
\justifying

\textit{Vascular Coordinate Common Framework} (VCCF) memungkinkan pemetaan setiap sel di ginjal, namun masih terdapat keterbatasan dalam pengetahuan tentang mikrovaskular. Segmentasi mikrovaskular pada citra \textit{Whole Slide Image} (WSI) jaringan ginjal manusia merupakan tugas penting yang memiliki berbagai manfaat dalam studi tentang ginjal, termasuk diagnosis dan pengobatan penyakit. Penelitian ini mengusulkan metode \textit{Fully Convolutional Network} dengan arsitektur Attention U-net yang diharapkan dapat meningkatkan kinerja U-net dasar dalam melakukan segmentasi mikrovaskular. Metode ini akan dievaluasi menggunakan metrik \textit{Interception Over Union} (IoU) dan \textit{Dice Similarity Coefficient} (DSC) untuk mengukur akurasi dan konsistensinya.
%%pada abstrak bahasa Inggris, separator desimal koma

\bigskip
\noindent
\textbf{Kata kunci:} \textit{Attention U-net, Deep Learning, Dice Similarity Coefficient (DSC), Interception Over Union (IoU), Segmentasi Mikrovaskular}  % masukkan keyword
\end{abstractind}
